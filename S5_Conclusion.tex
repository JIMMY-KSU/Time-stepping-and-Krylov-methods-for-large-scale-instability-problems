\section{Conclusions and perspectives}
\label{sec: conclusion}

With the ever increasing computational power available (roughly 20 to 25\% increase annually) and the development of high-performances computing (HPC), investigating the properties of realistic very large-scale nonlinear dynamical systems has become reachable. In the field of fluid dynamics, computation of fixed points of two-dimensional flows and investigation of the spectral properties of the corresponding linearized Navier-Stokes operator are now routinely performed on workstations or even laptops. The traditional way to do so is to use a so-called \emph{matrix-forming} approach where the Jacobian matrix of the system is explicitly assembled, whether one aims at computing a fixed point of the equations using Newton-like methods or at computing its leading eigenvalues and eigenmodes characterizing the linear stability properties of the fixed point considered. It must be noted however that the memory capabilities of computers increase at a slower rate than their computational capabilities. As a consequence, while simulations of very large-scale systems can now be performed (see \cite{rasthofer2017large} where the three-dimensional Navier-Stokes equations have been discretized using $10^{13}$ cells), using a matrix-forming approach to compute fixed points and study the stability properties of such systems becomes rapidly intractable. This gap between CPU and memory performances sprung the development of a new class of algorithms known as \emph{matrix-free}.

In this chapter, the reader has been introduced to a number of such matrix-free algorithms for the computation of fixed points and eigenpairs of the linearized operator. Most of these algorithms rely on the observation that existing simulation codes do not solve explicitly the continuous-time problem
$$\dot{\mathbf{x}} = \mathbfcal{F}(\mathbf{x})$$
but rather its discrete-time counterpart
$$\mathbf{x}_{k+1} = \mathbfcal{G}(\mathbf{x}_k).$$
Moreover, these time-stepper simulation codes do not form explicitly the matrices but only require to be able to compute their applications onto a given set of vectors. Given this observation, only minor modifications of existing time-stepper codes are required as to transform them into black- functions evaluating matrix-vector products, hence enabling practitioners to wrap them into powerful matrix-free iterative fixed points and/or eigenvalues solvers. Once again in the field of fluid dynamics, such an approach proved successful and allowed \cite{jfm:ilak:2012, jfm:loiseau:2014, pof:citro:2015, jfm:bucci:2018} to investigate the stability properties of the fixed points of nonlinear dynamical systems characterized by almost 50 millions degrees of freedom.

Time-stepper matrix-free approaches nonetheless suffer from a number of limitations and drawbacks. First and foremost, these approaches rely onto an existing simulation code to emulate the matrix-vector products required. As a consequence, the overall performances of the fixed points and eigenvalue solvers presented herein are essentially dictated by the efficiency of the time-stepper code considered. Second, a number of analyses (non-modal stability, receptivity, sensitivity, ...) may require the definition of an adjoint. While such adjoint operator simply reduces to the transconjugate operation in matrix-forming approaches, a dedicated adjoint time-stepper solver needs to be developed within the matrix-free framework. Although this may be quite challenging if the system is defined by a complicated set of nonlinear partial differential equations, one must note that recent developments in automatic differentiation might prove helpful (see the software TAPENADE \cite{TapenadeRef13} for instance). Finally, because the eigenvalue solvers described herein rely on Krylov techniques, one must bear in mind that only the leading subset of eigenpairs of the Jacobian matrix can be accurately computed.

Despite these limitations, time-stepper matrix-free approaches offer a practical and efficient computing framework for the investigation of very large-scale nonlinear dynamical systems. Provided one has access to an efficient time-stepper solver, their relative ease of implementation make the approaches described in the present chapter a standard choice of tools whenever matrix-forming approaches are intractable. Finally, these approaches can easily be combined with finite-differences approximation of the Jacobian matrix-vector product whenever a linearized time-stepper solver is not available, hence proving extremely versatile and applicable a very broad class of systems.
