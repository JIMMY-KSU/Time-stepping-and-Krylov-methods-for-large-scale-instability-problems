\section{Theoretical framework}
\label{sec: theory}

  Our attention is focused on the characterization of very high-dimensional nonlinear dynamical systems resulting from the spatial discretization of partial differential equations, e.g.\ the Navier-Stokes equations. Such dynamical systems can be written as
  \begin{equation}
    \displaystyle \frac{\mathrm{d} {\bf x}}{\mathrm{d}t} = \mathcal{F}({\bf x}, \mu),
  \end{equation}
where ${\bf x}$ is the $n \times 1$ state vector of the system, $t$ is time, $\mu$ is a control parameter and $\mathcal{F} : \mathbb{R}^n \to \mathbb{R}^n $ is the nonlinear dynamics. In this section, the reader will be introduced to the concepts of fixed points and linear stability. Particular attention will be paid to \emph{modal} and \emph{non-modal stability}, two fundamental concepts that have become prevalent in fluid dynamics over the past two decades. Note that the concept of \emph{nonlinear optimal perturbation} is beyond the scope of the present contribution. For interested readers, please refer to the recent work by \cite{???} and references therein.


  %%%%%%%%%%%%%%%%%%%%%%%%%%%%%%%%
  %%%%%                      %%%%%
  %%%%%     FIXED POINTS     %%%%%
  %%%%%                      %%%%%
  %%%%%%%%%%%%%%%%%%%%%%%%%%%%%%%%

  \subsection{Fixed points}
  \label{subsec: theory-fixed points}





  %%%%%%%%%%%%%%%%%%%%%%%%%%%%%%%%%%%
  %%%%%                         %%%%%
  %%%%%     MODAL STABILITY     %%%%%
  %%%%%                         %%%%%
  %%%%%%%%%%%%%%%%%%%%%%%%%%%%%%%%%%%

  \subsection{Modal stability analysis}
  \label{subsec: theory-modal stability}




  %%%%%%%%%%%%%%%%%%%%%%%%%%%%%%%%%%%%%%%%%%%%%%%%
  %%%%%                                      %%%%%
  %%%%%     NON-MODAL STABILITY ANALYSIS     %%%%%
  %%%%%                                      %%%%%
  %%%%%%%%%%%%%%%%%%%%%%%%%%%%%%%%%%%%%%%%%%%%%%%%

  \subsection{Non-modal stability analysis}
  \label{subsec: theory-non-modal stability}

    %-----> Optimal perturbation.
    \subsubsection{Optimal perturbation}
    \label{subsubsec: theory-optimal perturbation}

      %--> A Rayleigh quotient problem.
      \paragraph{Formulation using Rayleigh quotient}

      % --> Lagrange multipliers.
      \paragraph{Formulation using Lagrange multipliers}

    %-----> Optimal forcing.
    \subsubsection{Resolvent analysis}
    \label{subsubsec: theory-resolvent}

        %--> A Rayleigh quotient problem.
        \paragraph{Formulation using Rayleigh quotient}

        % --> Lagrange multipliers.
        \paragraph{Formulation using Lagrange multipliers}
