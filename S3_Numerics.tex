\section{Numerical methods}
\label{sec: numerics}

In this section, different techniques will be presented to solve modal and non-modal stability problems for very large-scale dynamical systems. Such very large-scale systems typically arise from the spatial discretization of partial differential equations, e.g.\ the Navier-Stokes equations in fluid dynamics. Throughout this section, the two-dimensional shear-driven cavity flow at various Reynolds numbers will serve as an example. The same configuration as \cite{??} is considered. The dynamics of the flow are governed by
\begin{equation}
  \begin{aligned}
    \displaystyle \frac{\partial \mathbf{U}}{\partial t} + \left( \mathbf{U} \cdot \nabla \right) \mathbf{U} & = - \nabla P + \frac{1}{Re} \nabla^2 \mathbf{U} \\
    \nabla \cdot \mathbf{U} & = 0,
  \end{aligned}
  \label{eq: numerics -- Navier-Stokes equations}
\end{equation}
where $\mathbf{U}$ is the velocity field and $P$ is the pressure field. Figure \ref{fig: numerics -- shear-driven cavity flow} depicts a typical vorticity snapshot obtained from direct numerical simulation at a supercritical Reynolds number.

Given a fixed point $\mathbf{U}_b$ of the Navier-Stokes equations \eqref{eq: numerics -- Navier-Stokes equations}, the dynamics of an infinitesimal perturbation $\mathbf{u}$ evolving on top of it are governed by
\begin{equation}
  \begin{aligned}
    \displaystyle \frac{\partial \mathbf{u}}{\partial t} + \left( \mathbf{u} \cdot \nabla \right) \mathbf{U}_b  + \left( \mathbf{U}_b \cdot \nabla \right) \mathbf{u} & = - \nabla p + \frac{1}{Re} \nabla^2 \mathbf{u} \\
    \nabla \cdot \mathbf{u} & = 0.
  \end{aligned}
  \label{eq: numerics -- linearized Navier-Stokes equations}
\end{equation}
Once projected onto a divergence-free vector space, Eq. \eqref{eq: numerics -- linearized Navier-Stokes equations} can be formally written as
\begin{equation}
  \dot{\mathbf{u}} = \mathbfcal{A}\mathbf{u},
  \label{eq: numerics -- linearized Navier-Stokes equations bis}
\end{equation}
where $\mathbfcal{A}$ is the linearized Navier-Stokes operator. After being discretized in space, $\mathbfcal{A}$ is a $n \times n$ matrix. For our example, the computational domain is discretized using ??? grid points, resulting in a total of $2 \times ??$ degrees of freedom. From a practical point of view, explicitly assembling the resulting matrix $\mathbfcal{A}$ would require approximately ?? Gb. Investigating the stability properties of this two-dimensional flow would thus not be possible on a simple laptop at the moment despite the simplicity of the case considered. It has to be noted however that, given an initial condition $\mathbf{u}_0$, the analytical solution to Eq. \eqref{eq: numerics -- linearized Navier-Stokes equations bis} reads
\begin{equation}
  \mathbf{u}(T) = \exp \left( \mathbfcal{A}T \right) \mathbf{u}_0,
  \notag
\end{equation}
where $\mathbfcal{M} = \exp \left( \mathbfcal{A}T \right)$ is the exponential propagator introduced previously. Although assembling explicitly this matrix $\mathbfcal{M}$ is even harder than assembling $\mathbfcal{A}$, its application onto the vector $\mathbf{u}_0$ can easily be computed using a classical time-stepping code. Such a \emph{time-stepper} approach has been popularized by \cite{??}. In the rest of this section, the different algorithms proposed for fixed point computation, linear stability and non-modal stability analyses will heavily rely on this time-stepper strategy. They require only minor modifications of an existing time-stepping code to be put into use.

  %%%%%%%%%%%%%%%%%%%%%%%%%%%%%%%%%%%%%%%%%%%%%%%%%%%%%%%
  %%%%%                                             %%%%%
  %%%%%     KRYLOV METHODS FOR LINEAR EQUATIONS     %%%%%
  %%%%%                                             %%%%%
  %%%%%%%%%%%%%%%%%%%%%%%%%%%%%%%%%%%%%%%%%%%%%%%%%%%%%%%

  \subsection{Krylov methods for for solving linear systems}
  \label{subsubsec: theory -- krylov methods}



  %%%%%%%%%%%%%%%%%%%%%%%%%%%%%%%%
  %%%%%                      %%%%%
  %%%%%     FIXED POINTS     %%%%%
  %%%%%                      %%%%%
  %%%%%%%%%%%%%%%%%%%%%%%%%%%%%%%%

  \subsection{Fixed points computation}
  \label{subsec: numerics-fixed points computation}
    %-----> Selective frequency damping.
    \subsubsection{Selective Frequency Damping}

    %-----> Newton-Krylov method.
    \subsubsection{Newton-Krylov method}

    %-----> BoostConv.
    \subsubsection{BoostConv}




  %%%%%%%%%%%%%%%%%%%%%%%%%%%%%%%%%%%%%%%%%%%%
  %%%%%                                  %%%%%
  %%%%%     MODAL STABILITY ANALYSIS     %%%%%
  %%%%%                                  %%%%%
  %%%%%%%%%%%%%%%%%%%%%%%%%%%%%%%%%%%%%%%%%%%%

  \subsection{Linear stability and eigenvalue computation}

    %-----> Power Iteration.
    \subsubsection{Power Iteration method}

    %-----> Arnoldi decomposition.
    \subsubsection{Arnoldi decomposition}

    %-----> Krylov-Schur decomposition.
    \subsubsection{Krylov-Schur decomposition}




  %%%%%%%%%%%%%%%%%%%%%%%%%%%%%%%%%%%%%%%%%%%%%%%%
  %%%%%                                      %%%%%
  %%%%%     NON-MODAL STABILITY ANALYSIS     %%%%%
  %%%%%                                      %%%%%
  %%%%%%%%%%%%%%%%%%%%%%%%%%%%%%%%%%%%%%%%%%%%%%%%

  \subsection{Non-modal stability and singular value decomposition}

    %-----> Optimal perturbation.
    \subsubsection{Optimal perturbation analysis}

    %-----> Resolvent analysis.
    \subsubsection{Resolvent analysis}
