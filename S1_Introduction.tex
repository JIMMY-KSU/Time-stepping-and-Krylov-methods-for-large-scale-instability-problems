\section{Introduction}
\label{sec: introduction}

Simulation of very large-scale linear or non-linear systems is a critical issue in many scientific fields. Fluid dynamics is full of examples where accurate and efficient methods having a reasonable computational cost and memory footprint are required. The study of flow stability is no exception, especially when one is interested in flows where the degree of spatial inhomogeneity is more and more important (one, two or three inhomogeneous directions). Historically, hydrodynamic stability analysis has always evolved according to the progress of computers, but also with the development of increasingly efficient numerical methods.

Before the 1980s, only problems where the flow has a single inhomogeneous spatial direction (generally perpendicular to the advection direction) could be discussed. The first discretization methods used were naturally the spectral or spectral collocations methods \cite{PT83, P2002} which offer a reasonable trade-off between computational cost and resolvability. One of the earliest examples of using such a method for linear stability analysis purposes can be found in \cite{J70}. Approximately at the same time, computation of the eigenvalue spectrum for a 1D flow were carried out \cite{J71, GJ75, M1976}, most often by shooting methods or Newton method coupled with continuation methods \cite{M69, DD1969}. One needs to wait until the mid-70s before eigenvalue solvers based on QR or QZ decompositions \cite{BM1984, BM1987, M90} start to be used in the study of a broad class of flows \cite{O1970, GH1970, MZH85, M90}. With the increased computational power, the 1980s and especially the 90s are marked by the rapid development of these methods for flows of increasing complexity. Various libraries are developed, the most famous ones being LAPACK \cite{LAPACK99}, MKL \cite{intelMKL} and ARPACK \cite{LSY97}. These libraries incorporate many iterative algorithms allowing for the full or partial computation of the eigenspectrum for flows with two inhomogeneous spatial directions, see \cite{T2003,T2011} for a review.

Most of the work carried out during this period consisted of linearizing the governing equations, discretizing them using methods such as spectral methods, finite-differences or finite elements and eventually solving the resulting eigenproblem often with an Arnoldi algorithm \cite{A1951,NO1993,LSY97}. The constant increase in geometric complexity of the flows addressed eventually led to a reformulation of the stability analyses and to the integration of these methods into existing simulation codes (e.g.\ FreeFem++ \cite{H2012}, Nek5000 \cite{FKMTLK08} or Nektar++ \cite{KS2005}). This evolution led to the increase importance of the numerical part (which was initially of theoretical nature). A glaring example of the weight of the numerics and resolution methods for very large-scale nonlinear dynamical systems can be illustrated in the computation of base flows, fixed point of the governing equations, which, unlike parallel and geometrically simple flows, can no longer be analytically obtained or simply approximated. Accurate computation of these equilibrium solutions is thus necessary. Fixed points solvers such as the selective frequency damping \cite{pof:akervik:2006}, Newton \cite{PW1998} and quasi-Newton \cite{SB2002}, or more recently RPM (Recursive Projection Method) \cite{siam:shroff:1993} and Boostconv \cite{citro2017efficient} are now commonly used to compute these equilibrium solutions.

Regarding the computation of the eigenpairs of the linearized Navier-Stokes operator, different strategies have been proposed over the years. When one tries to compute the stability of a fully three-dimensional flow, the computation and the manipulation of the Jacobian operator is a key problem mainly related to its dimension, of the order $10^6$-$10^8$. In the literature, two major approaches have emerged. The first one, known as "matrix-forming", explicitly assembles the Jacobian matrix The advantage of such an approach is that it is simple to compute the adjoint operator, which in this case is the hermitian of the discrete Jacobian matrix. However, this approach currently runs into computational difficulties for three-dimensional flows. Indeed, eigenvalue solvers typically require the computation of the inverse of the Jacobian, whose computational cost becomes almost out of reach. In the second approach, called "matrix-free", the Jacobian matrix is not explicitly assembled. Instead, one only needs to be able to evaluate the matrix-vector product so as to generate a Krylov sequence from which the spectral properties of the Jacobian are approximated. This method has the advantage of making stability analyses of very large-scale problems doable. One of its major drawbacks however is that one needs to write the continuous adjoint equations if interested into receptivity, sensitivity or non-modal stability problems.

The aim of this chapter is to take the point of view of the latter approach and to describe the main principles for both modal and non-modal analyses within a matrix-free and time-stepper computational framework. In that aspect, it follows the works of \cite{tuckerman2000bifurcation} and \cite{dijkstra2014numerical}. The different algorithms enabling the computation of the fixed points and the analysis of their modal and non-modal stability properties will be presented in detail. Advantages and limitations of each method will also be presented and illustrated by simple examples. The second objective is to give the reader a guide on how to use the different methods in order to implement them into an existing CFD code. For that purpose, the chapter is organized as follows: first, the theoretical frameworks of fixed points computation and modal and non-modal stability analyses are presented. The other sections present the different algorithms one needs to use for such analyses, taking care to compare their performances and to illustrate them on representative cases. Finally, the chapter ends with a conclusion and perspectives highlighting the most recent evolution of these methods and their possible extensions to more complex dynamics, especially to very large-scale time-periodic nonlinear dynamical systems.
